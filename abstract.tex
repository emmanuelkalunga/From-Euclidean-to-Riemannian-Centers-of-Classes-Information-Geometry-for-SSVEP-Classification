Brain Computer Interfaces (BCI) based on electroencephalography (EEG) rely on multichannel brain signal processing. Most of the state-of-the-art approaches deal with covariance matrices, and indeed Riemannian geometry has provided a substantial framework for developing new algorithms. Most notably, a straightforward algorithm such as Minimum Distance to Mean (MDM) yields competitive results when applied with a Riemannian distance or divergence. This applicative contribution aims at assessing the impact of several distances/divergences on real EEG dataset, as the invariances embedded in those distances/divergences have an influence on the classification accuracy. Riemannian centers of classes compare favorably with respect to Euclidean ones both in term of quality of results and of computational load. Riemannian distances cope with signal variabilities and reduce the adverse effect of artifacts in EEG signal.