%%%%%%%%%%%%%%%%%%%%%%%%%%%%%%%%%%%%%%%%%%%%%%%%%%%%%%%%%%%%%%%%%%%%%%%%%%%%%%%%

\section{Conclusion}
\label{sec:conclusion}

Riemannian approaches have been successfully applied on EEG signals for brain computer interfaces. 
Straightforward algorithms, such as Minimum Distance to Mean, provide competitive results with state-of-the-art methods, without requiring meticulous parametrization or optimization.
Working on covariance matrices in Riemannian spaces offers a wide choice of distances, embedding desirable invariances: it is thus possible to avoid the computation of user-specific spatial filters which are sensitive to artifacts and outliers.
Nonetheless, the estimation of the Riemannian geometric mean has a strong impact on the classifier accuracy.
This study investigates the performance of several distances and divergence on a real EEG dataset in the context of BCI based on the SSVEP paradigm.
The experimental results indicate that the $\alpha$-divergence yields the best accuracy after the selection of the best $\alpha$ values, but the Bhattacharyya distance has the lowest computational cost while providing honorable accuracies.

