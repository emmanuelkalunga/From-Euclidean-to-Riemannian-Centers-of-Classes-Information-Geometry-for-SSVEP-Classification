\section{Experimental results}
\label{sec:expresults}
This section presents experimental results obtained applying Euclidean and Riemannian distances in SSVEP classification task. 
%The performances are compared with LDA. 
The first part of this section describes the data used and the second part provides the assessment of the classification for the considered distances and divergences. %classifications accuracies obtained. 

\subsection{SSVEP dataset}
The experimental study is conducted on multichannel EEG signals recorded during an SSVEP-based BCI experiment~\cite{kalunga_hybrid_2014}. 
The subject sits in an electric wheelchair, his right upper limb is resting on the exoskeleton. 
The exoskeleton is functional but is not used during the recording of this experiment.
A panel of size 20x30 cm is attached on the left side of the chair, with 3 groups of 4 LEDs blinking at different frequencies as shown on Fig.~\ref{fig:ssvep}-(a)}. 
Even if the panel is on the left side, the user could see it without moving its head. 
The subjects were asked to sit comfortably in the wheelchair and to follow the auditory instructions, they could move and blink freely.
A sequence of trials is proposed to the user. 
A trial begin by an audio cue indicating which LED to focus on, or to focus on a fixation point set at an equal distance from all LEDs for the reject class. 

The subjects are presented with $\dF=$ 3 visual target stimuli blinking respectively at 13Hz, 21Hz and 17Hz.
It is a $\dK=$ 4 classes setup combining $\dF=$ 3 stimulus classes and one resting class (no-SSVEP).
In a session, 32 trials are recorded: 8 for each visual stimulus and 8 for the resting class. 
A trial is 4 second long. 
There are 12 subjects and the number of sessions recorded per subject varies from 2 to 5.
For each subject, a test set is made of 32 trials while the remaining trials (which might vary from 32 to 128) make up for the training set.

%A trial lasts 5 seconds and there is a 3 second pause between each trial. 
%The evaluation is conducted during a session consisting of 32 trials, with 8 trials for each frequency (13Hz, 17Hz and 21Hz) and 8 trials for the reject class, i.e. when the subject is not focusing on any specific blinking LED.
%TODO : mieux detailler l'experience et faire reference a la Figure (a)!

The EEG was recorded by a g.Mobilab+ equipped with $\dc=$ 8 electrodes/channels placed according to the international 10-20 system (PO7, PO3, POz, PO4, PO8, O1, Oz, O2). The reference electrode was placed on the left earlobe and the ground electrode was placed on Fz. Electrode impedance did not exceed 10 kOhm. The EEG sampling rate was 256Hz, with a bandpass filtering between 0.5 and 100Hz, and no spatial referencing is performed (to avoid to decrease the rank of the matrix). An example of EEG recording is shown on Fig.~\ref{fig:ssvep}-(b).
