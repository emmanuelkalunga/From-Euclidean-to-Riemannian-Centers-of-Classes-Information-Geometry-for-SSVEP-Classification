\section{Experimental results}
\label{sec:expresults}
This section presents experimental results obtained applying Euclidean and Riemannian distances in SSVEP classification task. 
%The performances are compared with LDA. 
The first part of this section describes the data used and the second part provides the assessment of the classification for the considered distances and divergences. %classifications accuracies obtained. 

\subsection{SSVEP dataset}
The experimental study is conducted on multichannel EEG signals recorded during an SSVEP-based BCI experiment~\cite{kalunga_hybrid_2014}. TODO : mieux detailler l'experience et faire reference a la Figure (a)!
The EEG was recorded by a TODO (preciser le modele de casque) equipped with $\dc=$ 8 electrodes/channels placed according to the international 10-20 system (PO7, PO3, POz, PO4, PO8, O1, Oz, O2). The reference electrode was placed on TODO (preciser la reference utilisee). Electrode impedance did not exceed TODO kOhm. The EEG sampling rate was TODO and TODO (preciser les preprocessing (filtrage passe-haut?)). An example of EEG recording is shown on TODO : faire reference a la Figure (b)!

The subjects are presented with $\dF=$ 3 visual target stimuli blinking respectively at 13Hz, 21Hz and 17Hz.
It is a $\dK=$ 4 classes setup combining $\dF=$ 3 stimulus classes and one resting class (no-SSVEP).
In a session, 32 trials are recorded: 8 for each visual stimulus and 8 for the resting class. 
A trial is 4 second long. 
There are 12 subjects and the number of sessions recorded per subject varies from 2 to 5.
For each subject, a test set is made of 32 trials while the remaining trials (which might vary from 32 to 128) make up for the training set.
